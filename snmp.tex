\documentstyle[a4]{article}
\begin{document}

\section{The MIB Database}

The compiler creates a special flat file respresntation of the compiled
MIBs upon completion. This file is read by the MIB tree class upon 
instantiation. The file format has been designed to support both 
SNMPv1 and SNMPv2 SMI.

The file consists of a sequence of records with colon-separated fields.
Comments lines can be included by starting the comment line with a hash
character in the first column. Each record should appear on a single line,
or on multiple lines with each line but the last terminated with a 
backslash character. When reading the file, backslashes are removed
except in the description field where they are converted to newline
characters. Trailing fields which are not required or applicable
may be omitted.

Each record consists of the following fields:

\begin{itemize}
\item the name of the object.
\item the full object identifier of the object described by the record.
An object is either an object-identifier node (e.g. iso(1)) or an
object type node (e.g. ifTable(2)). In the former case, all of the
remaining fields are omitted.
\item the base type of the object. This is identified using one of the
following codes:
\begin{itemize}
\item T - table (i.e. sequence of);
\item t - table entry (i.e. sequence);
\item n - null;
\item o - octet string;
\item i - integer;
\item en - enumerated integer;
\item bs - bit string;
\item oi - object identifier;
\item na - network address;
\item ip - IP address;
\ite ns - NSAP address;
\item u32 - 32-bit unsigned integer;
\item c32 - 32-bit unsigned counter;
\item c64 - 64-bit unsigned counter;
\item g64 - 64-bit unsigned gauge;
\item g32 - 32-bit unsigned gauge;
\item tt - time ticks;
\item oq - opaque;
\end{itemize}
\item the accessibility of the object, indicated by one of the codes:
\begin{itemize}
\item n - no access;
\item r - read only;
\item rw - read/write;
\item w - write only;
\item rc - read/create.
\end{itemize}
\item the status of the object, indicated by one of the codes:
\begin{itemize}
\item c - current;
\item d - deprecated;
\item ob - obsolete;
\item m - mandatory;
\item op - optional.
\end{itemize}
\item supplementary information, which varies with the object type.
This can be:
\begin{itemize}
\item a list of table indices for tables;
\item a list of symbolic values ((name,value) pairs) for enumerations
and bits strings;
\item a minimum and maximum value for other integer-based types;
\item a minimum and maximum length for string-based types.
\end{itemize}
\item the description of the object.
\item the default value of the object;
\item the units used by the object;
\end{itemize}

\end{document}

